\selectlanguage{ngerman}
\Abstract[Zusammenfassung]{}
% In einem Satz
Diese Dissertation befasst sich mit Methoden zur automatisierten Extraktion qualitativ hochwertiger, strukturierter Daten aus wissenschaftlichen Publikationen.
% Motivation
Strukturierte Daten über wissenschaftliche Publikationen ermöglichen essenzielle Elemente des akademischen Alltags, wie beispielsweise akademische Suchdienste und bibliometrische Kennzahlen.
Die Erzeugung derartiger Daten erfordert die Extraktion von Informationen aus den natürlichsprachlichen Inhalten von Publikationen---ein Prozess, der komplex und fehleranfällig ist.
Bestehende Extraktionsmethoden, und somit auch die damit gewonnenen Daten selbst, weisen diverse Mängel auf. Das ist problematisch, da folglich Anwendungen und Forschung basierend auf den derzeit verfügbaren Daten in ihrer Anwendbarkeit und Validität beschränkt sind.

% Lücke
Unter den Mängeln derzeit verfügbarer Methoden und Daten sind drei Bereiche von besonderer Bedeutung im akademischen Kontext.
(1)~\emph{Zitiernetzwerke} sind ein essenzieller Teil wissenschaftlicher Literatur und spielen eine zentrale Rolle bei Trendanalysen und Empfehlungssystemen. Trotz ihrer Wichtigkeit sind Zitiernetzwerke viel verwendeter Datensätze hochgradig unvollständig.
(2)~\emph{Sprachabdeckung}: Wissenschaft ist ein globales und damit inhärent multilinguales Unterfangen. Trotz zunehmender Anerkennung dieses Umstandes sind wichtige akademische Plattformen, Methoden und Datensätze auf englischsprachige Publikationen beschränkt.
(3)~\emph{Forschungsartefakte}, wie beispielsweise Methoden und Datensätze, werden zunehmend wichtig, da Forschung mehr und mehr durch Daten und deren algorithmische Verarbeitung vorangetrieben wird. Feingranulare Daten über Forschungsartefakte können vielversprechende Anwendungen wie Facettensuche und automatisierte Replikation ermöglichen. Bestehende Extraktionsmethoden erfassen allerdings nur grobe Daten über Forschungsartefakte, die für derartige Anwendungen nicht ausreichen.

% Lösung
Wir adressieren diese Mängel durch die Entwicklung von Data-Mining- und In\-for\-ma\-tions\-ex\-trak\-tions-Methoden, welche die Erstellung maschinenlesbarer Publikationskorpora ermöglichen. Zusätzlich quantifizieren wir die damit erzielten Verbesserungen in der Datenqualität.
Die hierfür umgesetzten Forschungsbeiträge sind wie folgt.
Als Basis unserer Forschung entwickeln wir eine Methode zur Erstellung eines großen Korpus von verknüpften Volltextdokumenten auf Basis von \LaTeX{}-Quelldateien.
Durch Anwendung unserer Methode auf der Gesamtheit von arXiv.org, erstellen wir den ersten Korpus verknüpfter Publikationen mit umfangreicher Abdeckung in der Physik, Mathematik und Informatik.
Aufbauend auf diesem Korpus entwickeln wir Ansätze, die Fortschritte in allen der drei zuvor erwähnten Mängelbereichen erzielten.
(1)~Wir entwickeln Methoden für die Verknüpfung von Literaturreferenzen, die state-of-the-art Ergebnisse in der Vollständigkeit von Zitiernetzwerken erzielen.
Basierend hierauf setzen wir neue Analyseformen um.
% Wir entwickeln eine Methode zur Verknüpfung bibliographischer Referenzen, mit welcher wir state-of-the-art Zitiernetzwerkvollständigkeit erreichen.
(2)~Wir präsentieren ein Verfahren zur Identifikation sprachübergreifender Zitate, und führen damit die bisher größte Analyse dieser Art von Zitaten durch.
Durch unsere Analyse identifizieren wir Herausforderungen für die Integration nicht-Englischer Publikationen.
(3)~Wir entwickeln feingranulare In\-for\-ma\-tions\-ex\-trak\-tions-Methoden für Forschungsartefakte und deren Parameter. Unsere Ansätze erzielen bessere Ergebnisse als leistungsstarke Vergleichsmethoden, und ermöglichen in ihrer Verwendung neue Formen von Analysen und Anwendungen.

% Zusammenfassung
Zusammengenommen adressieren unsere Beiträge zentrale Mängel existierender Methoden zur Extraktion strukturierter Daten aus wissenschaftlichen Publikationen.
Durch den Einsatz unserer Methoden erzielen wir signifikante Verbesserungen im Hinblick auf Datenqualität.
Für jeden unserer Ansätze demonstrieren wir dessen Umsetzbarkeit und Vorteile durch Evaluation und Anwendung auf großen Datenmengen.
Die Ergebnisse unserer Arbeit haben bereits Verwendung in verschiedenen Teilen der Forschungsgemeinschaft gefunden, was deren Nutzen zusätzlich bestätigt.
