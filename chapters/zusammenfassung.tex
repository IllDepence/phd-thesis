\selectlanguage{ngerman}
\Abstract[Zusammenfassung]{}
% Motivation
Wissenschaftliche Publikationen werden von Menschen geschrieben und \emph{von Menschen gelesen}.
Essenzielle Elemente des akademisches Alltags setzen jedoch voraus, dass Publikationen \emph{von Computern verarbeitet} werden.
Beispiele hierfür stellen akademische Suchmaschinen und bibliometrische Kennzahlen dar, die unabdingbar für Literaturrecherche und Entscheidungsprozesse sind.
Die Überleitung eines Mediums das \emph{von Menschen gelesen} wird, in eine Form die \emph{von Computern verarbeitet} werden kann, ist non-trivial und fehleranfällig.

% Lücke
Bestehende Methoden zur Erstellung von Daten, die Publikationen repräsentieren, und folglich auch die damit erstellten Daten selbst, weisen diverse Mängel auf. Explizit zeigen sich Defizite hinsichtlich der Verknüpfung, Sprachabdeckung, und Repräsentationsgranularität von Dokumenten. Dementsprechend sind Anwendungen und Forschung basierend auf diesen Daten in ihrer Anwendbarkeit und Aussagekraft beschränkt.

% Lösung
Wir adressieren diese Mängel durch die Entwicklung von Data-Mining- und In\-for\-ma\-tions\-ex\-trak\-tions-Methoden, welche die Erstellung maschinenlesbarer Publikationskorpora ermöglichen, die vollständiger verknüpft, umfangreicher, und feingranularer sind als bisher. Erzielte Fortschritte in der Datenqualität quantifizieren wir entsprechend der Dimensionen Relevanz, Genauigkeit, Aktualität, Vergleichbarkeit, sowie Vollständigkeit, wobei wir in jeder dieser Verbesserungen verzeichnen.

Die hierfür umgesetzten Forschungsbeiträge sind wie folgt.
Als Basis unserer Forschung stellen wir eine Methode zur Erstellung eines großen Korpus von verknüpften Volltextdokumenten auf Basis von \LaTeX-Quelldateien vor.
Aufbauend auf diesem Korpus entwickeln wir Ansätze, die Fortschritte in drei Kernbereichen bringen.
(1) Wir erzielen Verbesserungen in der Vollständigkeit der Zitiernetzwerkes durch den Einsatz einer Blocking- und Matching-Methode, sowie Verbesserungen in der Feingranularität von Dokumentrepräsentationen.
(2) Wir verzeichnen Fortschritte in der Sprachabdeckung von Dokumentverknüpfungen durch die Identifizierung und Analyse von sprachübergreifender Zitate.
(3) Wir entwickeln In\-for\-ma\-tions\-ex\-trak\-tions-Methoden für feingranulare Repräsentationen von Forschungsartefakten und deren Parametern.

% summary
Overall, our approaches address key shortcomings of existing methods for the creation of data representing publications.
For each of our approaches, we demonstrate its viability and benefits through evaluations and practical large-scale applications.
Our methods have already been adopted in several parts of the research community, which further confirms their utility.
