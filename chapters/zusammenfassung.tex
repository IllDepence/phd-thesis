\selectlanguage{ngerman}
\Abstract[Zusammenfassung]{}
% In einem Satz
Diese Dissertation befasst sich mit Methoden zur automatisierten Extraktion qualitativ hochwertiger, strukturierter Daten aus wissenschaftlichen Publikationen.
% Motivation
Strukturierte Daten über wissenschaftliche Publikationen ermöglichen Essenzielle Elemente des akademisches Alltags, wie beispielsweise akademische Suchdienste und bibliometrische Kennzahlen.
Die Erzeugung derartiger Daten erfordert die Extraktion von Informationen aus den natürlichsprachlichen Inhalten von Publikationen---ein Prozess, der komplex und fehleranfällig ist.
Bestehende Extraktionsmethoden, und somit auch die damit gewonnenen Daten selbst, weisen diverse Mängel auf. Das ist problematisch, da folglich Anwendungen und Forschung basierend auf den derzeit verfügbaren Daten in ihrer Anwendbarkeit und Validität beschränkt sind.

% Lücke
Unter den Mängeln derzeit verfügbarer Methoden und Daten sind drei Bereiche von besonderer Bedeutung im akademischen Kontext.
(1)~\emph{Zitiernetzwerke} sind 
(2)~\emph{Sprachabdeckung}: abseits englischsprachiger Publikationen und
(3)~\emph{Forschungsartefakte}

% Lösung
Wir adressieren diese Mängel durch die Entwicklung von Data-Mining- und In\-for\-ma\-tions\-ex\-trak\-tions-Methoden, welche die Erstellung maschinenlesbarer Publikationskorpora ermöglichen, die vollständiger verknüpft, umfangreicher, und feingranularer sind als bisher verfügbare. Erzielte Fortschritte hinsichtlich Datenqualität quantifizieren wir entsprechend der Dimensionen Relevanz, Genauigkeit, Aktualität, Vergleichbarkeit, sowie Vollständigkeit, wobei wir in jeder dieser Verbesserungen verzeichnen.

Die hierfür umgesetzten Forschungsbeiträge sind wie folgt.
Als Basis unserer Forschung stellen wir eine Methode zur Erstellung eines großen Korpus von verknüpften Volltextdokumenten auf Basis von \LaTeX-Quelldateien vor.
Aufbauend auf diesem Korpus entwickeln wir Ansätze, die Fortschritte in drei Kernbereichen bringen.
(1) Wir erzielen Verbesserungen in der Vollständigkeit der Zitiernetzwerkes durch den Einsatz einer Blocking- und Matching-Methode, sowie Verbesserungen in der Feingranularität von Dokumentrepräsentationen.
(2) Wir verzeichnen Fortschritte in der Sprachabdeckung von Dokumentverknüpfungen durch die Identifizierung und Analyse von sprachübergreifender Zitate.
(3) Wir entwickeln In\-for\-ma\-tions\-ex\-trak\-tions-Methoden für feingranulare Repräsentationen von Forschungsartefakten und deren Parameter.

% Zusammenfassung
Zusammengenommen adressieren unsere Beiträge zentrale Mängel existierender Methoden zur Erstellung maschinenlesbarer Publikationsrepresentationen.
Für jeden unserer Ansätze demonstrieren wir dessen Umsetzbarkeit und Vorteile durch Evaluation und Anwendung auf großen Datenmengen.
Die Ergebnisse unserer Arbeit haben bereits Verwendung in verschiedenen Teilen der Forschungsgemeinschaft gefunden, was deren Nutzen zusätzlich bestätigt.
