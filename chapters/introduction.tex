\chapter{Introduction}
\label{chp:introduction}

% establish the nature and role of scientific publications
Scientific publications are the \emph{discourse medium} and \emph{literary footprint} of academic progress. As such, they play a key role in the everyday workings of adademia.
% segue to the digial
Historically, the role of \emph{discourse medium} manifested itlsef phyically---through ink on paper. In time, digital methods of authoring %/composition?
and distribution enabled more efficient ways of dissemination.
Today, scientific publications are predominantly authored, distributed, consumed, and archived digitally~\cite{Lamers2018}. However, the form of todays digital publications still retains numerous and deep traces of its phyical ancestry.
% - b/c of discprepancy between current form and what would be even better for digital disseminatio and use (e.g. more structured), structured derivates of publications that enable digital services are of limited quality
% - historical baggage in part b/c still meant for human consumption
% - gap(?) between “pure” digital form and current form hinders more efficient use in digital srevices surrounding publications and academia


% What do I want to say and convey here?
% - scientific publications are the ``footprint'' and ``medium of discourse'' of academic progress
% - structured representation
% - 
% - 
% - 
% - 




\section{Motivation}

% What do I want to say and convey here?
% - why do we have/need scholarly data?
% - 

%Measuring the Evolution of a Scientific Field through Citation Frames~~\cite{Jurgens2018} (usable here (or in HyperPIE chapter) for extra justification of focus on parameters of used artifacts)

\begin{infobox-pub}
\textbf{Publication} \fullcite{Saier2020}
\end{infobox-pub}

\begin{infobox-discussion}
\textbf{Discussion} This text should show what a printed text will look like at this place. If you read this text, you will get no information. Really? Is there no information?
\end{infobox-discussion}

\begin{infobox-info}
\textbf{Remark} This text should show what a printed text will look like at this place. If you read this text, you will get no information. Really? Is there no information?
\end{infobox-info}

\begin{infobox-q}
\textbf{RQ1} How can the completeness of the citation network in scholarly data sets be improved?
\end{infobox-q}

\begin{infobox-progress}
      \begin{tabular}{ccl}
        \toprule
        Crit.\tnote{a} & Res.\tnote{b} & Explanation \\
        \midrule
        \textbf{C1} & {\large\textbf{+}} & Representative coverage in physics, mathematics, CS \\
        \textbf{C2} & $\circ$ & Primary focus on text content \\
        \textbf{C3} & {\large\textbf{+}} & $>96$\% accuracy in reference matching \\
        \textbf{C4} & {\large\textbf{+}} & Low noise due to using \LaTeX\ as data source \\
        \textbf{C5} & {\large\textbf{+}} & Publications until end of most recent full year \\
        \textbf{C6} & {\large\textbf{+}} & Provides MAG and arXiv IDs; DOIs in linked MAG \\
        \textbf{C7} & $\circ$ & Not considered at this stage \\
        \textbf{C8} & {\large\textbf{+}} & 42.6\% reference matching success rate \\
        \textbf{C9} & {\large\textbf{+}} & Full-text included \\
        \bottomrule
      \end{tabular}
\end{infobox-progress}

\section{Research Questions}

Research questions are provided as a guidance to the reader, giving a high-level perspective on the insights provided by the publications making up this dissertation.
% RQ "formulas":
% Q: "How can <goal> be achieved? A: "<proposed method>"
% Q: "How can <proposed method aspect> be leveraged to achieve <goal>? A: "<proposed method>"
% Q: "What's the nature of <analysis object>?" A: "<analysis result>

\section{Contributions}

\begin{table}[tb]
\centering
  \caption{Overview of publications reused in this dissertation.}
  \label{tab:primarypublicationoverview}
  \begin{tabular}{cllllclr}
    \hline
    \ & \ & \ & \ & \ & Author & Venue & \ \\
    Chap. & Venue & Year & Type & Length & Position & Rating & Ref. \\
    \hline
    3 & Scientometrics & 2020 & Journal & Full & 1 of 2 & SJR Q1 & \cite{Saier2020} \\
    \arrayrulecolor{lightgrey}\cline{1-8}
    \multirow{2}{*}{4} & JCDL & 2022 & Workshop & Full & 1 of 3 & Core A* & \cite{Saier2022ULITE} \\
    \ & JCDL & 2023 & Conference & Short & 1 of 3 & Core A* & \cite{Saier2023unarXive} \\
    \arrayrulecolor{lightgrey}\cline{1-8}
    \multirow{2}{*}{5} & ICADL & 2020 & Conference & Full & 1 of 2 & Core A & \cite{Saier2020xling} \\
    \ & IJDL & 2022 & Journal & Full & 1 of 3 & SJR Q2 & \cite{Saier2021} \\
    \arrayrulecolor{lightgrey}\cline{1-8}\arrayrulecolor{black}
    6 & ECIR & 2023 & Conference & Full & 1 of 4 & Core A & \cite{Saier2023hyperpie} \\
    \hline
    \end{tabular}
\end{table}

The contributions in this dissertation have been published in peer-reviewed international conferences and journals. Table~\ref{tab:primarypublicationoverview} gives an overview of the publications and the chapters they make up. Venue ranks are taken from Core\refurl{http://portal.core.edu.au/conf-ranks/}{2023-10-12} in the case of conferences and from SJR\refurl{https://www.scimagojr.com/}{2023-10-12} in the case of journals.\footnote{The ranks shown are the rating for the respective publication year, or the most up-to-date ranking if the latter is not listed. For workshops, the rank of the conference at which the workshop is hosted is shown.} For each of the publications, detailed author contributions according to the Contributor Roles Taxonomy\refurl{https://credit.niso.org/}{2023-10-12} are listed at the end of the respective section.

\begin{table}[tb]
\centering
  \caption{Overview of secondary publications not reused in this dissertation.}
  \label{tab:secondarypublicationoverview}
  \begin{tabular}{llllclr}
    \hline
    \ & \ & \ & \ & Author & Venue & \ \\
    Venue & Year & Type & Length & Position & Rating & Ref. \\
    \hline
    ECIR & 2019 & Workshop & Full & 1 of 2 & Core A & \cite{Saier2019} \\
    ECIR & 2020 & Conference & Full & 1 of 3 & Core A & \cite{Saier2020a} \\
    NAACL & 2021 & Workshop & Short & 3 of 4 & Core A & \cite{Krause2021} \\
    AAAI & 2022 & Workshop & Full & 2 of 3 & Core A* & \cite{Shapiro2022} \\
    ECIR & 2022 & Workshop & Full & 4 of 5 & Core A & \cite{Faerber2022bir} \\
    JCDL & 2022 & Conference & Full & 3 of 3 & Core A* & \cite{Nishioka2022} \\
    JCDL & 2023 & Conference & Short & 1 of 3 & Core A* & \cite{Saier2023cocon} \\
    \hline
    \end{tabular}
\end{table}

Additional publications (co-)authored leading up to and during the research period which are not a direct part of this dissertation, but nevertheless informed the overall research trajectory, are listed in Table~\ref{tab:secondarypublicationoverview}. Especially \cite{Saier2019} and \cite{Krause2021}, which constitute the results of the master's thesis preceding the doctoral research period, paved the way for this dissertation.

\section{Outline}
\Blindtext[1]

% unarXive
% |  |  |
% |  |  v
% |  |  blocking
% |  v
% |  unarXive22
% v       |
% xling   v
% |      Hyperpie
% v
% xling+
