\chapter{Introduction}
\label{chp:introduction}

% What do I want to say and convey here?
% - scientific publications are the ``footprint'' and ``medium of discourse'' of academic progress
% - structured representation

% establish the nature and role of scientific publications
Scientific publications are the \emph{discourse medium} and \emph{literary footprint} of academic progress. As such, they play a vital role in the everyday workings and advancement of academia.
% segue to the digital
Historically, this role manifested itself physically---through ink on paper. In time, digital methods for authoring %/composition?
and distribution enabled more efficient ways of dissemination.
Today, scientific publications are predominantly authored, distributed, consumed, and archived digitally~\cite{Lamers2018}. However, the form of today's digital publications still retains numerous and deep traces of its physical ancestry.

% Digital dissemination makes the transfer of new knowledge into the research community faster and cheaper. Digital archives enable search technology and analytics to operate on our \emph{literary footprint} of past and present academic progress. In this way, digitization facilitates both faster progress in science, \emph{and} technology helping scientists to keep up with the increase in speed. However, today's form of digital publications is far from optimal for powering such technology. In fact, the form of today's digital publications poses significant challenges for 

% While there lies enormous potential in a digital record of science, there remain significant challenges to realizing this potential. A key cause of these challenges is the historical baggage of today's digital publications. This dissertation is in pursuit of tackling these challenges.

% There lies enormous potential in a digital record of science. But realizing this potential

The historical baggage manifesting itself in the PDF files we read, share, and author, poses significant challenges to realizing the full potential of a digital record of science. While digital services in academia such as search and recommendation do exist, they are powered not by publications themselves, but rather by more structured, derivative representations of publications. The same holds for analyses of publications. The creation of such structured representations %/derivates?
is challenging and error prone, leading to a risk of subpar services and erroneous analyses. This dissertation represents and effort to tackle this challenge and alleviate the quality of structured representations of scientific publications.

% (random alliteration: structured surrogates of academic archives)

% - b/c of discrepancy between current form and what would be even better for digital dissemination and use (e.g. more structured), structured derivates of publications that enable digital services are of limited quality
% - historical baggage in part b/c still meant for human consumption
% - gap(?) between “pure” digital form and current form hinders more efficient use in digital services surrounding publications and academia

\section{Motivation}

% What do I want to say and convey here?
% - why do we have/need scholarly data?
% - in contrast to intro paragraphs above, make *necessity* of scholarly data for mitigating information overload clear
% - also make clear why there was realistic potential in approaching the challenge

In the following, we discuss the importance of structured representations of scientific publications---from hereon referred to as ``scholarly data'' for the sake of brevity%
\footnote{Note that in other contexts the term ``scholarly data'' can have a broader meaning than just ``structured representations of scientific publications''. For example, it can also include data on research institutions or funding bodies, without necessitating the context of a publication.}%
---, and based on that derive the motivation of the research project.

% * in the olden days there were polymaths
% * nowadays an expert can't humanly keep up with their field
% * → we need assistance
% * → my research in on enabling the creating of such assistance

% * since ever, humans have sought shortcuts
% * in decision making in science: h-index etc. (not sure if this fits b/c I don't specifically address such metrics)

The digitization of academic publishing has made the transfer of new knowledge into the research community faster, thereby enabling an acceleration of scientific progress. A second driver of acceleration is the increase in research spending across the world~\cite{CRS2022,OECD2023}.
This increase in scientific progress, while first and foremost a positive development, brings with it a growing challenge for researchers to keep up with the literature. This problem is referred to as ``information overload''~\cite{Landhuis2016}.
Fortunately, the digitization of academic publishing not only lead to %/facilitated
an increase in the rate at which research results are being published. It also marks the inception of scholarly data, and thereby enabled search technology and analytics to operate on large collections of digitally archived publications.
In this way, the existence of digital representations of publications also provides, to some degree, a remedy for information overload. Efficient search and recommendation services, for example, can aid researchers in navigating the deluge of publications they're faced with.
Similarly, decision processes in academia, such as the evaluation of institutions or researchers, is enabled by scholarly data through performance indicators. In this way, it provides a means for hiring and funding decisions.
In other words, scholarly data is a vital resource for decision making in academia on the individual as well as organisational level.

The quality of decisions made based on scholarly data, naturally, hinges on the quality of the scholarly data itself. Missing citation links in the data, for example, might cause researchers to overlook relevant related work, or a funding body under-evaluating an institution. Recalling that the creation of scholarly data is a challenging and error prone process, it stands to reason that efforts to improve scholarly data quality are a worthwhile endeavor. Based on these considerations, the overarching objective pursued in this dissertation is the development of methods for creating high quality scholarly data.

% * not only is my research addressing the now pressing and increasing issue of an increasing rate of publication
% * it also tackles blind spots the have been left unaddressed for long (x-ling)

\section{Problem Statement}

% What do I want to say and convey here?
% - make approached problem more concrete
%   - hand-wave-y in above:
%       ``[digital services in academia] are powered not by publications themselves,
%         but rather by more structured, derivative representations of publications''
%     -> publications authored by humans for humans -> scholarly data secondary/derivate
%     => concretization: derive structured representation from publications
%   - ``high quality''
%     -> briefly touch upon what's laid out in foundation chapter
%     => concretization: high quality = fitness for use
% - (maybe even narrow down to reference focus, maybe *even* more to LaTeX based)
% - give example

\section{Challenges}

% What do I want to say and convey here?
% - (TODO: look at foundations part and see if challenges are laid out clearly already)

% big picture view on bridging the gap between historical baggage documents (print, scan, PDF, ...) and digital representation
% - large existing current effort
%       - document image dewarping\refurl{https://github.com/fh2019ustc/Awesome-Document-Image-Rectification}{2023-11-21}
%       - OCR
%       - PDF based scholarly data stuff


%Measuring the Evolution of a Scientific Field through Citation Frames~~\cite{Jurgens2018} (usable here (or in HyperPIE chapter) for extra justification of focus on parameters of used artifacts)

\begin{infobox-pub}
\textbf{Publication} \fullcite{Saier2020}
\end{infobox-pub}

\begin{infobox-discussion}
\textbf{Discussion} This text should show what a printed text will look like at this place. If you read this text, you will get no information. Really? Is there no information?
\end{infobox-discussion}

\begin{infobox-info}
\textbf{Remark} This text should show what a printed text will look like at this place. If you read this text, you will get no information. Really? Is there no information?
\end{infobox-info}

\begin{infobox-q}
\textbf{RQ1} How can the completeness of the citation network in scholarly data sets be improved?
\end{infobox-q}

\begin{infobox-progress}
      \begin{tabular}{ccl}
        \toprule
        Crit.\tnote{a} & Res.\tnote{b} & Explanation \\
        \midrule
        \textbf{C1} & {\large\textbf{+}} & Representative coverage in physics, mathematics, CS \\
        \textbf{C2} & $\circ$ & Primary focus on text content \\
        \textbf{C3} & {\large\textbf{+}} & $>96$\% accuracy in reference matching \\
        \textbf{C4} & {\large\textbf{+}} & Low noise due to using \LaTeX\ as data source \\
        \textbf{C5} & {\large\textbf{+}} & Publications until end of most recent full year \\
        \textbf{C6} & {\large\textbf{+}} & Provides MAG and arXiv IDs; DOIs in linked MAG \\
        \textbf{C7} & $\circ$ & Not considered at this stage \\
        \textbf{C8} & {\large\textbf{+}} & 42.6\% reference matching success rate \\
        \textbf{C9} & {\large\textbf{+}} & Full-text included \\
        \bottomrule
      \end{tabular}
\end{infobox-progress}

\section{Research Questions}

Research questions are provided as a guidance to the reader, giving a high-level perspective on the insights provided by the publications making up this dissertation.
% RQ "formulas":
% Q: "How can <goal> be achieved? A: "<proposed method>"
% Q: "How can <proposed method aspect> be leveraged to achieve <goal>? A: "<proposed method>"
% Q: "What's the nature of <analysis object>?" A: "<analysis result>

\section{Contributions}

\begin{table}[tb]
\centering
  \caption{Overview of publications reused in this dissertation.}
  \label{tab:primarypublicationoverview}
  \begin{tabular}{cllllclr}
    \hline
    \ & \ & \ & \ & \ & Author & Venue & \ \\
    Chap. & Venue & Year & Type & Length & Position & Rating & Ref. \\
    \hline
    3 & Scientometrics & 2020 & Journal & Full & 1 of 2 & SJR Q1 & \cite{Saier2020} \\
    \arrayrulecolor{lightgrey}\cline{1-8}
    \multirow{2}{*}{4} & JCDL & 2022 & Workshop & Full & 1 of 3 & Core A* & \cite{Saier2022ULITE} \\
    \ & JCDL & 2023 & Conference & Short & 1 of 3 & Core A* & \cite{Saier2023unarXive} \\
    \arrayrulecolor{lightgrey}\cline{1-8}
    \multirow{2}{*}{5} & ICADL & 2020 & Conference & Full & 1 of 2 & Core A & \cite{Saier2020xling} \\
    \ & IJDL & 2022 & Journal & Full & 1 of 3 & SJR Q2 & \cite{Saier2021} \\
    \arrayrulecolor{lightgrey}\cline{1-8}\arrayrulecolor{black}
    6 & ECIR & 2023 & Conference & Full & 1 of 4 & Core A & \cite{Saier2023hyperpie} \\
    \hline
    \end{tabular}
\end{table}

The contributions in this dissertation have been published in peer-reviewed international conferences and journals. Table~\ref{tab:primarypublicationoverview} gives an overview of the publications and the chapters they make up. Venue ranks are taken from Core\refurl{http://portal.core.edu.au/conf-ranks/}{2023-10-12} in the case of conferences and from SJR\refurl{https://www.scimagojr.com/}{2023-10-12} in the case of journals.\footnote{The ranks shown are the rating for the respective publication year, or the most up-to-date ranking if the latter is not listed. For workshops, the rank of the conference at which the workshop is hosted is shown.} For each of the publications, detailed author contributions according to the Contributor Roles Taxonomy\refurl{https://credit.niso.org/}{2023-10-12} are listed at the end of the respective section.

\begin{table}[tb]
\centering
  \caption{Overview of secondary publications not reused in this dissertation.}
  \label{tab:secondarypublicationoverview}
  \begin{tabular}{llllclr}
    \hline
    \ & \ & \ & \ & Author & Venue & \ \\
    Venue & Year & Type & Length & Position & Rating & Ref. \\
    \hline
    ECIR & 2019 & Workshop & Full & 1 of 2 & Core A & \cite{Saier2019} \\
    ECIR & 2020 & Conference & Full & 1 of 3 & Core A & \cite{Saier2020a} \\
    NAACL & 2021 & Workshop & Short & 3 of 4 & Core A & \cite{Krause2021} \\
    AAAI & 2022 & Workshop & Full & 2 of 3 & Core A* & \cite{Shapiro2022} \\
    ECIR & 2022 & Workshop & Full & 4 of 5 & Core A & \cite{Faerber2022bir} \\
    JCDL & 2022 & Conference & Full & 3 of 3 & Core A* & \cite{Nishioka2022} \\
    JCDL & 2023 & Conference & Short & 1 of 3 & Core A* & \cite{Saier2023cocon} \\
    \hline
    \end{tabular}
\end{table}

Additional publications (co-)authored leading up to and during the research period which are not a direct part of this dissertation, but nevertheless informed the overall research trajectory, are listed in Table~\ref{tab:secondarypublicationoverview}. Especially \cite{Saier2019} and \cite{Krause2021}, which constitute the results of the master's thesis preceding the doctoral research period, paved the way for this dissertation.

\section{Outline}
\Blindtext[1]

% unarXive
% |  |  |
% |  |  v
% |  |  blocking
% |  v
% |  unarXive22
% v       |
% xling   v
% |      Hyperpie
% v
% xling+
