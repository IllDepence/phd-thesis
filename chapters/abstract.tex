\selectlanguage{english}
\Abstract{}
% motivation
Data extracted from scientific publications serves as the foundation for vital infrastructure in academia.
%Vital infrastructure in academia relies on data extracted from publications.
For example, academic search engines and bibliometric indicators, which are indispensable for literature review and decision making, % are based on operate on / rely on / would not be possible without
% machine-readable data % information
% representing publications.
rely on it.
% For example, academic search engines and bibliometric indicators, which are indispensable for literature review and decision making, are based on % operate on / rely on / would not be possible without
However, the extraction of such data is a challenging and error-prone process.
This is because scientific publications are written for human readers, and therefore do not provide the structure and unambiguity necessary for automated processing. Faced with this challenge, existing extraction methods and the data they produce are limited in areas of key importance.

% gap
Specifically, we focus on three areas of limitation because of their significance to academia.
% 1, why important, how limited
Our first focus are documents' interconnections through citations.
Citation networks are a key characteristic of scientific corpora, and are vital for common use cases such as the calculation of bibliometric indicators and the development of citation-based recommender systems. Despite this importance, widely used data sets have highly incomplete citation networks, putting the validity of analysis results and effectiveness of systems in question.
% 2, why important, how limited
Second, we focus on language coverage.
Science is a global and therefore inherently multi-lingual endeavor. Accordingly, capturing the state and progress of science in data necessitates including publications written in various languages. However, despite a growing awareness in the
%natural language processing
research community, important platforms, approaches, and data sets in the scholarly domain are still limited to English publications only. % maybe add sentence why this is bad
% 3, why important, how limited
Our third focus area is the representation of research artifacts in data. ...

% why exactly there gaps?
% - citation network -> key characteristic that is vital for common use cases (bibliometric indicators, cit. based rec., ...)
% - language coverage -> growing awareness in NLP, scholarly data also affected
% - content representation granularity -> research increasingly driven by curated data and algorithmic processing



Specifically, there are key shortcomings regarding documents' interconnections, language coverage, and content representation granularity.
For example, widely used data sets have highly incomplete citation networks, are limited to publications written in English, or fail to accurately capture mathematical notation. As a consequence, applications and research building on this data are of limited scope and validity.

% solution
To address these issues, we present a set of data mining and information extraction approaches that enable the creation of machine-readable publication corpora more complete, extensive, and of finer granularity than previously available.
% TODO: make orthogonal nature of gap/focus vs. quality dimensions clearer
We quantify our contributions to better data quality along the dimensions relevance, accuracy, timeliness, comparability, and completeness, and achieve improvements across all of them.

In particular, we present the following.
As the foundation of our research, we introduce a method for creating a large-scale corpus of linked, full-text documents from publications' \LaTeX\ sources.
Utilizing the resulting corpus, we further present approaches yielding advances in three areas.
First, we demonstrate improvements in the completeness of citation networks though the use of a blocking and matching method, as well as improvements in the granularity of document representations.
Second, we show advances in the language coverage of document interconnections through identifying and analyzing cross-lingual citations.
Third, we present information extraction approaches for fine-granular representations of research artifacts and their parameters.

% summary
Overall, our approaches address key shortcomings of existing methods for the creation of data representing publications.
For each of our approaches, we demonstrate its viability and benefits through evaluations and practical large-scale applications.
Our methods have already been adopted in several parts of the research community, which further confirms their utility.
