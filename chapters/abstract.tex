\selectlanguage{english}
\Abstract{}
% in one simple* sentence   *I tried ...
This dissertation addresses the challenge of generating high-quality, machine-readable representations of scientific publications at a large scale.
% ease in from a more general point
Structured data representing scientific publications is the basis for vital infrastructure in academia, such as academic search and bibliometric performance indicators. %, which are indispensable for literature review and decision making.
Generating such data involves information extraction from publications' natural language content, which makes it a challenging and error-prone process.
% This is because scientific publications are written for human readers, and therefore do not provide the structure and unambiguity necessary for automated processing.
Existing extraction methods and the data they produce are limited in several ways. %. %areas of key importance.
This is problematic, because it means that applications and research based on currently available data are of limited scope and validity.

% gap
Among the limitations of currently available methods %for generating machine-readable representations of scientific publications
and data,
three areas are of particular importance due to their relevance in the academic context.
% 1, why important, how limited
(1)~\emph{Citation networks} are a key characteristic of scientific literature, and are vital for common use cases such as trend analyses and recommender systems. Despite this importance,
%widely used data sets %have highly incomplete citation networks are missing a large portion of citations contained in the publications they are based on.
% This puts the validity of analysis results and effectiveness of systems using the data in question.
citation networks of widely used data sets are highly incomplete.
% 2, why important, how limited
(2)~\emph{Language coverage}: science is a global and therefore inherently multi-lingual endeavor.
% Accordingly, capturing the state and progress of science in data necessitates including publications written in various languages. However, despite a growing awareness in the
%natural language processing
% research community,
Despite a growing awareness of this, % /Despite growing recognition of this
important platforms, approaches, and data sets in the scholarly domain are still limited to English publications only.
% 3, why important, how limited
(3)~\emph{Research artifacts}, such as methods and data sets, become more and more important, as science is increasingly driven by curated data and algorithmic processing. Fine-grained representations of research artifacts bear large potential for applications like faceted academic search and automated reproduction. However, existing extraction methods only yield shallow representations of research artifacts, not sufficient for these use cases.

% solution
To address these issues, we develop %a set of
data mining and information extraction approaches, that enable the creation of machine-readable publication corpora.
We furthermore quantify the improvements we achieve in terms of data quality in each area of limitation.
%
In particular, we make the following contributions.
As the foundation of our research, we develop a method for creating a large-scale corpus of interlinked, full-text documents from publications' \LaTeX{} sources.
Applying our method to all of arXiv.org, we create the first corpus of interlinked publications with extensive %representative
coverage in physics, mathematics, and computer science.  % justifiable to claim representative coverage?
Utilizing our corpus, we further present approaches yielding advances in all of the three aforementioned areas of limitation.
(1)~We develop a methodology for linking bibliographic references, which achieves state-of-the-art citation network completeness. Based on this, we perform novel types of citation analyses.
(2)~We present a method for identifying cross-lingual citations and, utilizing it, perform the largest analysis of this type of citation to date. Through our analysis, we are able to identify challenges for integrating non-English publications.
(3)~We develop information extraction approaches for fine-granular representations of research artifacts and their parameters.
Our methods achieve an improvement over strong baselines, and their utilization enables novel types of analyses and applications.

% In particular, we present the following.
% As the foundation of our research, we introduce a method for creating a large-scale corpus of linked, full-text documents from publications' \LaTeX{} sources.
% Utilizing the resulting corpus, we further present approaches yielding advances in three areas.
% First, we demonstrate improvements in the completeness of citation networks though the use of a blocking and matching method, as well as improvements in the granularity of document representations.
% Second, we show advances in the language coverage of document interconnections through identifying and analyzing cross-lingual citations.
% Third, we present information extraction approaches for fine-granular representations of research artifacts and their parameters.

% summary
Overall, our approaches address key shortcomings of existing methods for the creation of structured data representing publications.
Through their use, we achieve significant improvements in terms of data quality.
For each of our approaches, we demonstrate its viability and benefits through evaluations and practical large-scale applications.
Our methods have already been adopted in several parts of the research community, which further confirms their utility.
