\chapter{Conclusion}
\label{chp:conclusion}

\section{Summary}

This dissertation set out to to alleviate the state of scholarly data. In order to achieve this, the following research objective was set.

\begin{infobox-objective}
\textbf{Research Objective}\\
Develop an automated process that takes as input scientific publications, and produces as output a high-quality derivate representation of the publications suitable for digital processing.
\end{infobox-objective}

Criteria for high-quality were defined across the following dimensions.\footnote{See Table~\ref{tab:scholdataquali} in Chapter~\ref{chp:foundations} for the list of criteria.}

\begin{infobox-progress}
      \textbf{Data Quality Dimensions}\\
       (1)~relevance, (2)~accuracy, (3)~timeliness, (4)~comparability, (5)~completeness
\end{infobox-progress}

To achieve this, the following steps were performed. In Chapter~\ref{chp:corpus} we developed a corpus creation method transforming publications' \LaTeX\ source files into a large-scale corpus of of linked, full-text documents. The presented method also includes a highly accurate reference matching procedure. Applying our method on the complete set of all publications on arXiv.org, we created the data set \emph{unarXive}, which was used as a basis for all subsequent work. %``Corpus''
In Chapter~\ref{chp:covgran} we presented improvements regarding the citation network and the granularity of document representations. For the citation network, a blocking technique was developed that, applied on the set of references in a corpus, increases the number of matched references and bibliographic couplings. With an updated corpus creation method, the \emph{unarXive} data set achieved a more complete citation network. The updated procedure furthermore enabled more fine-granular document representations. %``Reference Coverage and Granularity''
In Chapter~\ref{chp:xling} we studied cross-lingual citations in the \emph{unarXive} corpus. For this, we found a method to reliably identify this type of citation based on raw reference strings. In our study, which is the largest of its kind to date, we analysed cross-lingual citations' prevalence, usage, and impact. %``References Across Languages''
Lastly, in Chapter~\ref{chp:params} we developed methods for extracting information about research artifacts and their usage parameters from publication full-texts. Applying our best performing method on \emph{unarXive}, we found difference in parameter reporting patterns across several disciplines. %``References with Usage Parameters''

The presented work makes the following contributions towards the research objective.

\begin{infobox-progress}
      \textbf{Scholarly Data Quality Contributions - Overview}\vspace{0.5em}

      \begin{tabular}{lccccc}
        \toprule
        Quality Dimension & Quality Criterion\tnote{b} & \multicolumn{4}{c}{Contribution} \\
        \midrule
        \multirow{2}{*}{Relevance} & \textbf{C1} & {\large\textbf{+}} & = & {\large\textbf{+}} & $\circ$ \\
         & \textbf{C2} & $\circ$ & {\large\textbf{+}} & $\circ$ & {\large\textbf{+}} \\
        \arrayrulecolor{lightgrey}\hline\arrayrulecolor{black}
        \multirow{2}{*}{Accuracy} & \textbf{C3} & {\large\textbf{+}} & = & $\circ$ & $\circ$ \\
         & \textbf{C4} & {\large\textbf{+}} & = & $\circ$ & $\circ$ \\
        \arrayrulecolor{lightgrey}\hline\arrayrulecolor{black}
        Timeliness & \textbf{C5} & {\large\textbf{+}} & {\large\textbf{+}} & $\circ$ & $\circ$ \\
        \arrayrulecolor{lightgrey}\hline\arrayrulecolor{black}
        \multirow{2}{*}{Comparability} & \textbf{C6} & {\large\textbf{+}} & {\large\textbf{+}} & {\large\textbf{+}} & $\circ$ \\
         & \textbf{C7} & $\circ$ & {\large\textbf{+}} & $\circ$ & {\large\textbf{+}} \\
        \arrayrulecolor{lightgrey}\hline\arrayrulecolor{black}
        \multirow{2}{*}{Completeness} & \textbf{C8} & {\large\textbf{+}} & {\large\textbf{+}} & $\circ$ & $\circ$ \\
         & \textbf{C9} & {\large\textbf{+}} & {\large\textbf{+}} & $\circ$ & $\circ$ \\
        \midrule
        \midrule
        \multicolumn{2}{r}{Chapter} & \ref{chp:corpus} & \ref{chp:covgran} & \ref{chp:xling} & \ref{chp:params} \\
        \multicolumn{2}{r}{Publication} & \cite{Saier2020} & \cite{Saier2022ULITE,Saier2023unarXive} & \cite{Saier2020xling,Saier2021} & \cite{Saier2023hyperpie} \\
        \bottomrule
      \end{tabular}

      \vspace{0.5em}
      \begin{footnotesize}
      \textbf{Legend}\\
      \textbf{+}: SOTA/improvement/etc. (see respective chapter)\\
      =: equal to previous\\
      $\circ$: not considered.
      \end{footnotesize}
\end{infobox-progress}

\section{Conclusion}

Consider giving an overview of the impact the presented work had so far (see colloquium presentation).


\section{Outlook}
Always look on the bright side of life.
